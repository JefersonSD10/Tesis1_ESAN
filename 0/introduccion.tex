%La línea de abajo es para quitar encabezado
%\thispagestyle{plain}

\chapter*{Introducción}
\markboth{Introducción}{Introducción}
\addcontentsline{toc}{chapter}{Introducción}

Por muchos años, en especial en las dos últimas décadas, diversos proyectos emprendedores han sido lanzados en distintas plataformas web, buscando un objetivo compartido por todos: ser financiados en un determinado plazo para hacer realidad estas ideas. Entre fracasos y éxitos, han surgido nuevas tendencias, así como nuevas perspectivas de estudios de estos casos para encontrar la clave que descifre las variables de éxito.

A diferencia de estudios previos, en donde casi la totalidad de antecedentes engloba todas las categorías existentes en Kickstarter, uno de los sitios web de crowdfunding más populares, la principal motivación para realizar esta investigación fue de proponer un nuevo enfoque para resolver el problema de predecir si un proyecto en esta plataforma será financiado o no durante su campaña, considerando solamente aquellos de la categoría Tecnología, cuyo ratio de 20\% de éxito representa la más baja de todas. Bajo este escenario, se buscó poder desarrollar un modelo predictivo con estas condiciones iniciales desfavorables sin que se afecte su desempeño por el comportamiento de otras categorías con mejores ratios de éxito. El nuevo aporte consistió en diseñar un modelo de Aprendizaje Multimodal Profundo utilizando información de las principales variables cuantitativas y el contenido textual de la campaña, tanto la descripción del proyecto redactada por el creador como los comentarios recibidos por los patrocinadores acerca del mismo, es decir, la interacción social directa entre los stakeholders.

En el Capítulo I, se describe la realidad problemática de la investigación y el entorno en que se desenvuelve. Asimismo, se formulan los problemas, objetivos, hipótesis, justificación y delimitación del estudio.

En el Capítulo II, se detallan los antecedentes principales que fueron considerados al estudiar el problema de clasificación desde distintas perspectivas y estrategias. A continuación, se expone la base teórica en donde son abordados los conceptos técnicos que fueron aplicados, desde los fundamentos de la Inteligencia Artificial hasta métodos que forman parte de ella. El capítulo cierra con el marco conceptual en donde se explican términos relacionados al financiamiento colectivo y cómo funciona su entorno.

En el Capítulo III, se describe el diseño, tipo, enfoque, población, muestra y la operacionalización de las variables de la investigación. Luego se expone la metodología de implementación de la solución, desde el criterio de su elección hasta la puesta en marcha de las fases y actividades incurridas, así como los entregables comprometidos. Después, se explaya la metodología para la medición de resultados de la implementación. Finalmente, el capítulo concluye con el detalle del cronograma de actividades y presupuesto.

En el Capítulo IV, se detalla el desarrollo de la solución, desde cada modelo construido según la modalidad correspondiente hasta el modelo apilado final, y las tareas ejecutadas que se plantearon efectuar en el anterior capítulo.

En el Capítulo V, se analizan y discuten los resultados obtenidos de los experimentos, desde el tiempo de ejecución hasta los valores calculados por cada métrica de clasificación aplicada. Aquí también se comparan los resultados del modelo propuesto con la línea de base.

En el Capítulo VI, se comentan las conclusiones principales de toda la investigación y se realizan recomendaciones en base a las fortalezas, oportunidades de mejora y trabajos que, en el futuro, podrían ser desarrollados.

La investigación concluye con las referencias utilizadas en el trabajo y los anexos que complementan con mayor información dentro de cada enunciado citado.
