%La línea de abajo es para quitar encabezado
%\thispagestyle{plain}

\chapter*{Introducción}
\markboth{Introducción}{Introducción}
\addcontentsline{toc}{chapter}{Introducción}

1

2

El primer capítulo aborda el tema de la investigación y su contexto. Además, se discuten los objetivos del estudio, sus hipótesis, justificación y delimitación.

En el segundo capítulo, varios enfoques y puntos de vista examinan los antecedentes principales del problema de la insuficiente cobertura de red inalámbrica. Además, se proporciona la base teórica que aborda los conceptos técnicos aplicados, desde los fundamentos de la Inteligencia Artificial hasta las técnicas utilizadas en su desarrollo. El marco conceptual del capítulo concluye con una explicación de términos relacionados con la cobertura y el funcionamiento de las redes inalámbricas.

El tercer capítulo detalla el diseño, tipo, enfoque, población, muestra y operacionalización de las variables de la investigación. Luego se explican los criterios de selección, la ejecución de fases y actividades y los entregables comprometidos. Posteriormente, se presenta el método para evaluar los resultados de la implementación. Finalmente, el capítulo termina con predicciones y cobertura detalladas de Aps Indoor.

El cuarto capítulo describe el desarrollo de la solución, incluyendo todos los modelos construidos según la modalidad correspondiente hasta el modelo apilado final, así como las tareas planteadas en el capítulo anterior.

Los resultados de los experimentos se analizan y discuten en el quinto capítulo, que incluye el tiempo de ejecución y los valores de cada métrica de predicción aplicada. Además, se analiza el desempeño del modelo propuesto en comparación con la línea base.

Los principales hallazgos de la investigación se discuten en el sexto capítulo, así como las fortalezas, las oportunidades de mejora y los trabajos futuros.

Las referencias utilizadas y los anexos de la investigación proporcionan información adicional sobre cada afirmación mencionada.