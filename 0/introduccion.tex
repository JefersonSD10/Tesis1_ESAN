%La línea de abajo es para quitar encabezado
%\thispagestyle{plain}

\chapter*{Introducción}
\markboth{Introducción}{Introducción}
\addcontentsline{toc}{chapter}{Introducción}

En los últimos años, ha habido un creciente interés en mejorar la cobertura de redes inalámbricas en ambientes internos mediante el uso de Inteligencia Artificial Generativa para anticipar las ubicaciones más apropiadas para los puntos de acceso (APs). Esta tendencia ha surgido como respuesta a la necesidad de perfeccionar la calidad de la conectividad inalámbrica en edificios de varios pisos, donde las condiciones de propagación de la señal presentan mayor complejidad y variabilidad.

En contraste con estudios previos que se ocupaban de optimizar la cobertura inalámbrica de manera generalizada, este análisis se concentra de manera específica en entornos internos y en la predicción de ubicaciones particulares para los APs. La razón principal detrás de esto es la relevancia de conseguir una cobertura uniforme y de alta calidad dentro de estructuras complejas, donde la disposición estratégica de los APs puede marcar una diferencia considerable en cuanto al rendimiento de la red y la satisfacción del usuario.

El primer capítulo aborda el tema de la investigación y su contexto. Además, se discuten los objetivos del estudio, sus hipótesis, justificación y delimitación.

En el segundo capítulo, varios enfoques y puntos de vista examinan los antecedentes principales del problema de la insuficiente cobertura de red inalámbrica. Además, se proporciona la base teórica que aborda los conceptos técnicos aplicados, desde los principios de la IA hasta las técnicas utilizadas en su desarrollo. El marco conceptual del capítulo concluye con una explicación de términos relacionados con la cobertura y el funcionamiento de las redes inalámbricas.

El tercer capítulo detalla el enfoque, población, muestra, diseño, tipo y operacionalización de las variables estudiadas. Luego se explican los criterios de selección, la ejecución de fases y actividades y los entregables comprometidos. Posteriormente, se presenta el método para evaluar los resultados de aplicación. Por último, se concluye con predicciones y cobertura detalladas de Aps Indoor.

El cuarto capítulo describe el desarrollo de la solución, incluyendo todos los modelos construidos, así como lo planteado en el capítulo anterior.

Los resultados de los experimentos se analizan y discuten en el quinto capítulo, que incluye los valores de las métricas y el tiempo de ejecución de la predicción utlizada. Además, se analiza el desempeño del modelo propuesto en comparación con la línea base.

Los principales hallazgos de la investigación se discuten en el sexto capítulo, así como las posibles mejoras, las fortalezas y las investigaciones futuras.

Las referencias utilizadas y los anexos de la investigación proporcionan información adicional sobre cada afirmación mencionada.