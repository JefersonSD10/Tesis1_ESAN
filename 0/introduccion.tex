%La línea de abajo es para quitar encabezado
%\thispagestyle{plain}

\chapter*{Introducción}
\markboth{Introducción}{Introducción}
\addcontentsline{toc}{chapter}{Introducción}

1

2

El Capítulo I describe la realidad problemática de la investigación y su entorno. Además, se formulan los problemas del estudio, sus objetivos, hipótesis, justificación y delimitación.


El Capítulo II analiza los antecedentes principales del problema de la falta de cobertura de red inalámbrica desde una variedad de puntos de vista y enfoques. A continuación, se presenta la base teórica en la que se abordan los conceptos técnicos que se aplicaron, desde los principios fundamentales de la Inteligencia Artificial hasta los métodos que se utilizan en su desarrollo. El marco conceptual del capítulo termina explicando términos relacionados con la cobertura de redes inalámbrica y cómo funciona su entorno.

En el Capítulo III, se describen el diseño, el tipo, el enfoque, la población, la muestra y la operacionalización de las variables de la investigación. Luego se explica la metodología de implementación de la solución, desde el criterio de elección hasta la puesta en marcha de las fases y actividades, así como los entregables comprometidos. Posteriormente, se presenta la estrategia para medir los resultados de la implementación. Finalmente, el capítulo termina con la predicción y cobertura detallada de Aps Indoor.

El Capítulo IV, describe cómo se desarrolló la solución, desde cada modelo construido según la modalidad correspondiente hasta el modelo apilado final, junto con las tareas planteadas para realizar en el capítulo anterior.

En el Capítulo V, se analizan y discuten los resultados de los experimentos, desde el tiempo de ejecución hasta los valores calculados por cada métrica de predicción aplicada. Además, se hace una comparación entre los resultados del modelo propuesto y la línea de base.

El Capítulo VI analiza los hallazgos principales de la investigación y hace sugerencias sobre fortalezas, oportunidades de mejora y trabajos futuros.

La investigación concluye con las referencias utilizadas en el trabajo y los anexos, que proporcionan más información sobre cada afirmación mencionada.